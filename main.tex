\documentclass[
	12pt, % Default font size, values between 10pt-12pt are allowed
	%letterpaper, % Uncomment for US letter paper size
	%spanish, % Uncomment for Spanish
]{fphw}

% Template-specific packages
\usepackage[american]{babel}
\usepackage{inputenc} % Required for inputting international characters
\usepackage[T1]{fontenc} % Output font encoding for international characters
\usepackage{mathpazo} % Use the Palatino font
\usepackage[scheme=plain]{ctex}
\usepackage{graphicx,subfigure} % Required for including images
\usepackage{booktabs} % Required for better horizontal rules in tables
\usepackage{listings} % Required for insertion of code
\usepackage[framed,numbered,autolinebreaks,useliterate]{mcode} %matlab code
\usepackage{fontspec}
\usepackage{enumerate} % To modify the enumerate environment
\usepackage{amsmath}
\usepackage{amssymb}
\usepackage{mathptmx}
\usepackage{bm}
%----------------------------------------------------------------------------------------
\setmainfont{Times New Roman} % 设置英文字体为Times New Roman,可自选

% Code-Style 代码模板
\lstdefinestyle{mystyle}{
    basicstyle=\ttfamily\footnotesize, % the font and size for code blocks
    breakatwhitespace=false,           % sets if automatic breaks should only happen at whitespace
    breaklines=true,                   % sets automatic line breaking
    keepspaces=true,                   % keeps spaces in text, useful for keeping indentation of code (possibly needs columns=flexible)
    numbers=left,                      % where to put the line-numbers; possible values are (none, left, right)
    numbersep=5pt,                     % how far the line-numbers are from the code
    showspaces=false,                  % show spaces everywhere adding particular underscores; it overrides 'showstringspaces'
    showstringspaces=false,            % underline spaces within strings only
    tabsize=4                          % sets default tabsize to 2 spaces
}
\lstset{style=mystyle}
% 使用方法:\lstinputlisting[language=Python]{code/a2.py}
%----------------------------------------------------------------------------------------
%	ASSIGNMENT INFORMATION
%----------------------------------------------------------------------------------------
\begin{document}
\title{Assignment \#1} % Assignment title
\author{Xiao Wang, 小王 12014689} % Student name and ID
\date{March 6, , 2024} % Due date

% ----------------------- This is where you put the logo at the top ---------------------
\begin{center}
    \includegraphics[scale=0.75]{img/logo.png}
\end{center}

%\institute{SOUTHERN UNIVERSITY\\OF SCIENCE AND TECHNOLOGY} % Institute or school name
\class{Introduction to Mathematical Logic } % Course or class name
\professor{Prof. Tao } % Professor or teacher in charge of the assignment

%----------------------------------------------------------------------------------------

\setlength{\abovecaptionskip}{-0.1cm}
\setlength{\belowcaptionskip}{0cm}   %调整图片标题与下文距离


\maketitle % Output the assignment title, created automatically using the information in the custom commands above

%----------------------------------------------------------------------------------------
%	ASSIGNMENT CONTENT
%----------------------------------------------------------------------------------------

\section*{Question 1}
\begin{problem}
	\includegraphics[width=440pt]{img/logo.png} % Problem image here
\end{problem}
%------------------------------------------------
\subsection*{Answer}
\begin{enumerate}[(\itshape a\normalfont)]
    \item From the Question 1(a), we have the $P(S_A)=0.8$ and $P(S_B)=0.6$, the $S_A$ and $S_B$ make decisions independently means that $P(S_AS_B)=P(S_A)P(S_B)$, So we have all the possible outcomes are 
\end{enumerate}
%----------------------------------------------------------------------------------------
\clearpage

\section*{Question 2}
\begin{problem}
	\includegraphics[width=440pt]{img/logo.png}
\end{problem}
%------------------------------------------------
\subsection*{Answer}
\begin{enumerate}[(\itshape a\normalfont)] % Sub-questions styled as italic letters
	\item From the question 2, we have the conditional mean $$E(Y|X=i)=\sum_{j=0}^i j\cdot P(Y=j|X=i)$$, and the
\end{enumerate}
%----------------------------------------------------------------------------------------
\clearpage

\section*{Question 3}
\begin{problem}
    \includegraphics[width=440pt]{img/logo.png}\\ % Mutiple images can be added here
    \includegraphics[width=440pt]{img/logo.png}
\end{problem}
%------------------------------------------------
\subsection*{Answer} 
\begin{enumerate}[(\itshape a\normalfont)]
    \item From sub-question (a), we can get the triangle $S=\{(x,y):-6<y<x<6\}$ showing in the fig 1. 
\end{enumerate}
%----------------------------------------------------------------------------------------
\clearpage

\section*{Question 4}
\begin{problem}
    \includegraphics[width=440pt]{img/logo.png}
\end{problem}
%------------------------------------------------
\subsection*{Answer}
\begin{enumerate}[(\itshape a\normalfont)]
    \item \lstinputlisting[language=Java]{code/demo.java} % The first code block 
\end{enumerate}
%----------------------------------------------------------------------------------------

\end{document}
